% !TEX root=../master.tex
\chapter{Introduction}
\label{ch:introduction}

\section{Research Motivation}
\label{sec:intro_motivation}
The concept of synergy is well understood and applied throughout our society. Everywhere we are involved in groups, teams, committees, and organizations, all with the intent of utilizing combined skills and talent to accomplish greater tasks than we could do on our own. With the recent growth in robotics and autonomous vehicle research, it is important to recognize that these same concepts of synergy apply to autonomous systems as well. While it is important to continue to develop and expand the capability of individual robots, we need to teach robots to work together without human intervention.

To illustrate the power of using a multi-agent system over a single vehicle, consider the following search and rescue scenario: A group of hikers has been reported missing in a treacherous mountain range with limited satellite connectivity. While a ground based rescue mission would take too long and the use of a helicopter might be dangerous, a group of synergistic UAS (Unmanned Aerial Systems) would prove useful in locating the group and providing necessary communication and supplies as quickly and efficiently as possible. Using a team of UAS would allow for greater ability to carry out a prolonged search, more effective surveillance of the area, and more capability to provide supplies which could be too heavy or numerous for an individual UAS to handle. If a similar mission were attempted with a single vehicle, it is likely that the search would take much longer, requiring the vehicle to pause and resume searching multiple times to recharge. Once the group was found, the rescue vehicle would also be limited in the weight, shape, and amount of supplies it could provide.

In this example scenario, the multi-agent system is superior to a single vehicle because it allows the team of UAS to overcome some of their primary limitations including a relative short battery life and a low payload capacity. In addition to search and rescue, there are many applications that would greatly benefit from the extended capacity of effective multi-agent systems, including wildlife surveillance, forest fire monitoring, construction management, disaster relief, and medical support.

% My research has focused on developing technologies that will allow the replacement of single UAS with robust multi-agent solutions.
Because the full multi-agent problem is far beyond the scope of my graduate research, I have focused my efforts on developing solutions to two main problems:
\begin{enumerate}
    \item Mitigating battery limitations through robust track correlation of moving targets.
    \item Enabling more robust counter-UAS solutions using a coordinated multi-agent defense strategy.
\end{enumerate}
\stcomment{Finish up here?}

\section{Background}
\label{sec:intro_background}
